\documentclass[12pt]{article}
\usepackage{amsmath, amssymb, amsthm}
\usepackage{geometry}
\usepackage{graphicx}
\usepackage{hyperref}

% Page layout
\geometry{letterpaper, margin=1in}

% Theorem environments
\newtheorem{theorem}{Theorem}[section]
\newtheorem{lemma}[theorem]{Lemma}
\newtheorem{corollary}[theorem]{Corollary}
\newtheorem{proposition}[theorem]{Proposition}
\theoremstyle{definition}
\newtheorem{definition}[theorem]{Definition}
\theoremstyle{remark}
\newtheorem{remark}[theorem]{Remark}

% Commands
\newcommand{\Qn}{\text{Q}_n}

% Title and author
\title{Title of the Paper}
\author{Author Name}
\date{\today}

\begin{document}

\maketitle

\begin{abstract}
This is the abstract of the paper. It provides a brief summary of the content.
\end{abstract}

\section{Introduction}

\section{Definition}


\section{Main Results}

\begin{theorem}
    Let $w \in \Qn$ then the parking of $w$ is also in $\Qn$.
\end{theorem}

\begin{proof}
    Consider $i \in [n]$. We can write $w$ as $w = w(1)w(2) \ldots i \ldots j \ldots j \ldots i \ldots w(2n)$. Notice that the
    first $i$ must park in ahead of the first $j$. If this was not the case then $j$ must be lucky while $i$ was not lucky; however, 
    since $i$ appears before $j$, the $j$th spot would be open for $i$ to park in which is a contradiction. Thus, the first $i$ must 
    park before the first $j$. The second pair will always be unlucky; hence, would maintain their relative order in the parking. This 
    shows that $w$ is in $\Qn$.
\end{proof}

It turns out that the set of image of $\Qn$ under the parking map is a subset of the set of stirling permutations that is 
easily characterized.

\begin{lemma}
    The image of $\Qn$ under the parking map is the set of stirling permutations such that $w(j) = i$ implies $i \leq j$.
\end{lemma}

\begin{proof}
    Since a car with prefence $i$ can only park in position $i$ or later and a parked stirling permutation is a stirling permutation, 
    we must have $w(j) = i$ parking at or after spot $i$. 
\end{proof}

In fact, we have $|p(\Qn)| = n!$ because 

\subsection{Theorem and Proof}
\begin{theorem}
Statement of the theorem.
\end{theorem}

\begin{proof}
Proof of the theorem.
\end{proof}

\section{Conclusion}
The conclusion section goes here.

\begin{thebibliography}{9}
\bibitem{example} Author, \textit{Title of the Book}, Publisher, Year.
\end{thebibliography}

\end{document}